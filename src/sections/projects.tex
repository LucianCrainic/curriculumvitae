\sectionTitle{Projects}{\faLaptop}

\begin{projects}
  \project
  {Brain Cancer Segmentation and Classification}{}
  {\github{davidebelcastro-sig/BrainCancerSegmentation-Classification} \website{https://makerfairerome.eu/it/espositori/?edition=2023&exhibit=2320497}{Maker Faire 2023}}
  {This project aims to use \textbf{Machine Learning} and \textbf{AI} to segment and classify brain tumors.
    It includes a user-friendly \textbf{GUI} for uploading and visualizing images, leveraging various libraries for its
    implementation. The tool provides segmentation results, including tumor probability and area, and classifies
  tumors into glioma, meningioma, or pituitary categories.}
  {Python, OpenCV, Scikit-Learn, TensorFlow, NumPy, Flet}

  \project
  {Jarvis: Personal Moodle Exam Assistant}{}
  {\github{struggling-student/Jarvis}}
  {An \textbf{automated tool} developed using Python and Selenium, designed to answer questions on the \textit{Moodle platform}. The tool can identify and answer questions that it has previously encountered. If it encounters new questions during a quiz, it has the capability to download and store them for future use. This ensures that over time, it becomes more knowledgeable about the questions in the quizzes.}
  {Python, Selenium, OpenCV, Scikit-Learn, Pillow}

  \project
  {Tiny-ViT: Lightweight Vision Transformer for Efficient Image Recognition}{}
  {\github{struggling-student/Tiny-ViT}}
  {Developed a compact \textbf{Vision Transformer (ViT)} model optimized for efficiency, balancing performance and computational cost. Implemented and trained Tiny-ViT for \textit{image classification tasks}, leveraging model pruning and quantization techniques to enhance deployment on edge devices.}
  {Computer Vision, Deep Learning, PyTorch, TensorFlow, Scikit-Learn}
\end{projects}
